\documentclass{article}

\usepackage{fancyhdr}
\usepackage{extramarks}
\usepackage{amsfonts}
\usepackage{syntax}
\usepackage{stmaryrd}
\usepackage{mathpartir}

%
% Basic Document Settings
%

\topmargin=-0.45in
\evensidemargin=0in
\oddsidemargin=0in
\textwidth=6.5in
\textheight=9.0in
\headsep=0.25in

\linespread{1.1}

\pagestyle{fancy}
\chead{NV Grammar}
\lfoot{\lastxmark}
\cfoot{\thepage}

\renewcommand\headrulewidth{0.4pt}
\renewcommand\footrulewidth{0.4pt}

\setlength\parindent{30pt}

\newcommand{\Z}{\mathbb{Z}}
\newcommand{\Zt}{$\Z$}

% Create a relational rule
% [#1] - Additional mathpartir arguments
% {#2} - Name of the rule
% {#3} - Premises for the rule
% {#4} - Conclusions for the rule
\newcommand{\relationRule}[4][]{\inferrule*[lab={\sc #2},#1]{#3}{#4}}

\newcommand{\rel}[1]{\ensuremath{\llbracket {#1} \rrbracket}}
\newcommand{\ttt}{\texttt}


\begin{document}
Before evaluating this relation, we are given a map type $\tau_m = \texttt{dict}[\tau_k][\tau_v]$ and a set of literals $\{k_1, \dots, k_n\}$ of type $\tau_k$ representing possible keys into maps of type $\tau_m$. For now we will assume that this set contains \emph{all} keys that are used in the program; we will probably be able to relax this assumption later. We also assume that we have run type inference, so we can determine the type of an expression during evaluation.

This relation transforms all values of type $\tau_m$ to the tuple type $\tau_t = ((\tau_k, \tau_v), \dots (\tau_k, \tau_v))$ composed of $n$ pairs. Note that we don't need to keep track of the default value, since it will be assigned during construction; however, if we ever extend \ttt{filter} to work on non-set types,
we will probably need to track the default as well.

%Notes: filter can only be used on sets, and uses the default value of false.
%map cannot change the type of the map, but otherwise works as expected. It also changes the default value.
%mapif p f m changes values whose key satisfies p, and leaves other values unchanged.
%combine f m1 m2 creates a new map whose value for key k is f m1[k] m2[k]

\section*{Values}

\begin{mathpar}
	\relationRule{identifer}{
		\null
	}{
		\rel{x} = x
	}
\end{mathpar}

\begin{mathpar}
	\relationRule{literal - bool}{
		b : \texttt{bool}
	}{
		\rel{b} = b
	}
\end{mathpar}

\begin{mathpar}
	\relationRule{literal - int}{
		n \mbox{ has an integer type} 
	}{
		\rel{n} = n
	}
\end{mathpar}

\begin{mathpar}
	\relationRule{literal - none}{
		\null		
	}{
		\rel{\texttt{None}} = \ttt{None}
	}
\end{mathpar}

\begin{mathpar}
	\relationRule{Some}{
		\null
	}{
		\rel{\texttt{Some}\ e} = \ttt{Some} \rel{e}
	}
\end{mathpar}


\begin{mathpar}
	\relationRule{Tuple}{
		\null
	}{
		\rel{(e_1, e_2, \dots, e_n)} = (\rel{e_1}, \rel{e_2}, \dots, \rel{e_n})
	}
\end{mathpar}

\begin{mathpar}
	\relationRule{Function}{
		\null
	}{
		\rel{\texttt{fun}\ x_1, \dots, x_n \to e} = \ttt{fun}\ x_1, \dots, x_n \to \rel{e}
	}
\end{mathpar}

\begin{mathpar}
	\relationRule{Emptyset - not type $\tau_m$}{
		e = \{\} \\
		e \not: \tau_m
	}{
		\rel{e} = e
	}
\end{mathpar}

\begin{mathpar}
	\relationRule{Emptyset - type $\tau_m$}{
		e = \{\} \\
		e : \tau_m
	}{
		\rel{e} = ((k_1, false), (k_2, false), \dots, (k_n, false))
	}
\end{mathpar}

\section*{Non-map Expressions}

\begin{mathpar}
	\relationRule{Not}{
		\null
	}{
		\rel{!e_1} = !\rel{e_1}	
	}
\end{mathpar}
	
\begin{mathpar}
\relationRule{Binop}{
	 \oplus \mbox{ is a binary operator}
   }{
   	\rel{e_1 \oplus e_2} = \rel{e_1} \oplus \rel{e_2}	
   }
\end{mathpar}

\begin{mathpar}
	\relationRule{Function application}{
		\null
	}{
		\rel{e_1 e_2} = \rel{e_1} \rel{e_2}	
	}
\end{mathpar}

\begin{mathpar}
	\relationRule{Let-binding}{
		\null
	}{
		\rel{\ttt{let}\ p = e_1\ \ttt{in}\ e_2} = \ttt{let}\ p = \rel{e_1}\ \ttt{in}\ \rel{e_2}	
	}
\end{mathpar}

\begin{mathpar}
	\relationRule{Match}{
		\null
	}{
		\rel{\ttt{match}\ e\ \ttt{with}\ |\ p_1 \to e_1 \dots |\ p_k \to e_k} = 
		\ttt{match}\ \rel{e}\ \ttt{with}\ |\ p_1 \to \rel{e_1} \dots |\ p_k \to \rel{e_k}
	}
\end{mathpar}

\begin{mathpar}
	\relationRule{If-else}{
		\null
	}{
		\rel{\ttt{if}\ e_1\ \ttt{then}\ e_2\ \ttt{else}\ e_3} = 
		\ttt{if}\ \rel{e_1}\ \ttt{then}\ \rel{e_2}\ \ttt{else}\ \rel{e_3}
	}
\end{mathpar}

\section*{Declarations}

\begin{mathpar}
	\relationRule{Let declaration}{
		\null
	}{
		\rel{\ttt{let}\ x = e} = \ttt{let}\ x = \rel{e}
	}
\end{mathpar}

\begin{mathpar}
	\relationRule{Attribute type - not $\tau_m$}{
		\tau \neq \tau_m
	}{
		\rel{\ttt{type attribute} = \tau} = \ttt{type attribute} = \tau
	}
\end{mathpar}


\begin{mathpar}
	\relationRule{Attribute type - $\tau_m$}{
		\null
	}{
		\rel{\ttt{type attribute} = \tau_m} = \ttt{type attribute} = \tau_t
	}
\end{mathpar}

\begin{mathpar}
	\relationRule{Symbolic type - not $\tau_m$}{
		\tau \neq \tau_m
	}{
		\rel{\ttt{symbolic}\ x : \tau} = \ttt{symbolic}\ x : \tau
	}
\end{mathpar}

\begin{mathpar}
	\relationRule{Symbolic type - $\tau_m$}{
		\null
	}{
		\rel{\ttt{symbolic}\ x : \tau_m} = \ttt{symbolic}\ x : \tau_t
	}
\end{mathpar}

\begin{mathpar}
	\relationRule{Symbolic value}{
		\null
	}{
		\rel{\ttt{symbolic}\ x = e} = \ttt{symbolic}\ x = \rel{e}
	}
\end{mathpar}

\begin{mathpar}
	\relationRule{Require}{
		\null
	}{
		\rel{\ttt{require}\ e} = \ttt{require}\ \rel{e}
	}
\end{mathpar}

\begin{mathpar}
	\relationRule{Node declaration}{
		\null
	}{
		\rel{\ttt{let nodes} = n} = \ttt{let nodes}\ = n
	}
\end{mathpar}

\begin{mathpar}
	\relationRule{Edge declaration}{
		\null
	}{
		\rel{\ttt{let edges} = lst} = \ttt{let edges}\ = lst
	}
\end{mathpar}

\section*{Map expressions}
Each of the following expressions have at least one subexpression which must be a map. The following rules apply if and only if those subexpressions have type $\tau_m$. If not, we recurse as usual but don't otherwise change the expression.

I'm noting that some of these rules involve a lot of duplication, so we might want to look into translations
which e.g. store duplicated expressions in a local variable first.

They also involve a lot of extra unpacking since we don't have syntax for getting/setting with tuples. It
would probably be good to extend the map syntax to work on tuples as well, so long as that doesn't mess
anything up; we could also create a new syntax for tuples.

\begin{mathpar}
	\relationRule{CreateDict}{
		e' = \rel{e}
	}{
		\rel{\ttt{createDict}\ e} = ((k_1, e'), (k_2, e'), \dots, (k_n, e'))	
	}
\end{mathpar}

\begin{mathpar}
	\relationRule{Get}{
		k = k_i \\
		x_i \mbox{ fresh}
	}{
		\rel{m[k]} = 
		\ttt{let}\ ((\_, \_), \dots, (\_, x_i), \dots, (\_, \_)) = \rel{m}\ \ttt{in}\ 
		x_i
	}
\end{mathpar}

\begin{mathpar}
	\relationRule{Set}{
		k = k_i \\
		x_1, \dots, x_n \mbox{ fresh}
	}{
		\rel{m[k := e]} = 
		\ttt{let}\ ((\_, x_1), \dots, (\_, x_n)) = \rel{m}\ \ttt{in}\ 
		((k_1, x_1), \dots, (k_i, \rel{e}), \dots, (k_n, x_n))
	}
\end{mathpar}

\begin{mathpar}
	\relationRule{Map}{
		f' = \rel{f} \\
		x_1, \dots, x_n \mbox{ fresh}
	}{
		\rel{\ttt{map}\ f\ m} = 
		\ttt{let}\ ((\_, x_1), \dots, (\_, x_n)) = \rel{m}\ \ttt{in}\ 
		((k_1, f'\ x_1), \dots, (k_n, f'\ x_n))
	}
\end{mathpar}

\begin{mathpar}
	\relationRule{MapIf}{
		f' = \ttt{fun}\ k\ v\ \to \ttt{if}\ p\ k\ \ttt{then}\ (k, \rel{f}\ v)\ \ttt{else}\ (k, v) \\
		x_1, \dots, x_n \mbox{ fresh}
	}{
		\rel{\ttt{mapIf}\ p\ f\ m} = 
		\ttt{let}\ ((\_, x_1), \dots, (\_, x_n)) = \rel{m}\ \ttt{in}\ 
		(f'\ (k_1, x_1), \dots, f'\ (k_n, x_n))
	}
\end{mathpar}

\begin{mathpar}
	\relationRule{Filter}{
		p' = \ttt{fun}\ k\ v\ \to \ttt{if}\ p\ k\ \ttt{then}\ (k, v)\ \ttt{else}\ (k, \ttt{false}) \\
		x_1, \dots, x_n \mbox{ fresh}
	}{
		\rel{\ttt{filter}\ p\ m} = 
		\ttt{let}\ ((\_, x_1), \dots, (\_, x_n)) = \rel{m}\ \ttt{in}\ 
		(p'\ (k_1, x_1), \dots, p'\ (k_n, x_n))
	}
\end{mathpar}

\begin{mathpar}
	\relationRule{Combine}{
		f' = \rel{f} \\
		x_1, \dots, x_n, y_1, \dots y_n \mbox{ fresh}
	}{
		\rel{\ttt{combine}\ f\ m_1\ m_2} = 
		\ttt{let}\ ((\_, x_1), \dots, (\_, x_n)) = \rel{m_1}\ \ttt{in}\ 
		\ttt{let}\ ((\_, y_1), \dots, (\_, y_n)) = \rel{m_2}\ \ttt{in}\ \\ 
		((k_1, f'\ x_1\ y_1), \dots, (k_n, f'\ x_n\ y_n))
	}
\end{mathpar}

\begin{mathpar}
	\relationRule{Union}{
		f' = \rel{f} \\
		x_1, \dots, x_n, y_1, \dots y_n \mbox{ fresh}
	}{
		\rel{m_1\ \ttt{union}\ m_2} = 
		\ttt{let}\ ((\_, x_1), \dots, (\_, x_n)) = \rel{m_1}\ \ttt{in}\ 
		\ttt{let}\ ((\_, y_1), \dots, (\_, y_n)) = \rel{m_2}\ \ttt{in}\ \\ 
		((k_1, x_1\ \|\ y_1), \dots, (k_n, x_n\ \|\ y_n))
	}
\end{mathpar}


\begin{mathpar}
	\relationRule{Intersection}{
		f' = \rel{f} \\
		x_1, \dots, x_n, y_1, \dots y_n \mbox{ fresh}
	}{
		\rel{m_1\ \ttt{inter}\ m_2} = 
		\ttt{let}\ ((\_, x_1), \dots, (\_, x_n)) = \rel{m_1}\ \ttt{in}\ 
		\ttt{let}\ ((\_, y_1), \dots, (\_, y_n)) = \rel{m_2}\ \ttt{in}\ \\ 
		((k_1, x_1\ \&\&\ y_1), \dots, (k_n, x_n\ \&\&\ y_n))
	}
\end{mathpar}

\end{document}