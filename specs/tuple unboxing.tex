\documentclass{article}

\usepackage{fancyhdr}
\usepackage{extramarks}
\usepackage{amsfonts}
\usepackage{syntax}
\usepackage{stmaryrd}
\usepackage{mathpartir}

%
% Basic Document Settings
%

\topmargin=-0.45in
\evensidemargin=0in
\oddsidemargin=0in
\textwidth=6.5in
\textheight=9.0in
\headsep=0.25in

\linespread{1.1}

\pagestyle{fancy}
\chead{Tuple Unboxing}
\lfoot{\lastxmark}
\cfoot{\thepage}

\renewcommand\headrulewidth{0.4pt}
\renewcommand\footrulewidth{0.4pt}

\setlength\parindent{30pt}

\newcommand{\Z}{\mathbb{Z}}
\newcommand{\Zt}{$\Z$}

% Create a relational rule
% [#1] - Additional mathpartir arguments
% {#2} - Name of the rule
% {#3} - Premises for the rule
% {#4} - Conclusions for the rule
\newcommand{\relationRule}[4][]{\inferrule*[lab={\sc #2},#1]{#3}{#4}}

\newcommand{\rel}[1]{\ensuremath{\llbracket {#1} \rrbracket}}
\newcommand{\ttt}{\texttt}
\newcommand{\transform}{\rightsquigarrow}
\newcommand{\proj}{\pi}
\newcommand{\ttuple}{\tau_1 * \ldots * \tau_n}
\newcommand{\etuple}{(e_1,\ldots,e_n)}


\begin{document}
This document defines a transformation $\transform$, such that every
expression with a tuple type is a tuple expression. The following invariant
should hold after this transformation:
$$ \Gamma \vdash e : \ttuple \rightarrow e = \etuple$$

For defining tuple unboxing we add a tuple projection construct
in the language $\proj_i$.

\section*{Values}

\begin{mathpar}
	\relationRule{identifer - tuple}{
		\null
	}{
		\Gamma \vdash x : \ttuple \transform (\proj_1 x, \ldots, \proj_n x)
	}
\end{mathpar}

\begin{mathpar}
	\relationRule{identifer}{
		\tau \neq \ttuple
	}{
		\Gamma \vdash x : \tau \transform x
	}
\end{mathpar}

\begin{mathpar}
	\relationRule{literal}{
		\tau \in \{\texttt{bool}, \texttt{int}\}
	}{
          \Gamma \vdash n : \tau \transform n
	}
\end{mathpar}

\begin{mathpar}
	\relationRule{Function}{
          e \transforms e'
	}{
		\Gamma \vdash \texttt{fun}\ x \to e : \tau_1 \to \tau_2 \transforms \ttt{fun}\ x \to e'
	}
\end{mathpar}

\section*{Expressions}


\begin{mathpar}
	\relationRule{Tuple}{
		\null
	}{
          \Gamma \vdash \etuple : \tau \transform \etuple
	}
\end{mathpar}

\begin{mathpar}
	\relationRule{Not}{
          e \transforms e'
	}{
		\Gamma \vdash !e : \tau \transforms !e'	
	}
\end{mathpar}
	
\begin{mathpar}
\relationRule{Binop}{
	 \oplus \mbox{ is a binary operator} \\ e_1 \transforms e_1' \\ e_2 \transforms e_2'
   }{
   	\Gamma \vdash e_1 \oplus e_2 : \tau \transforms e_1' \oplus e_2'
   }
\end{mathpar}

\begin{mathpar}
	\relationRule{Function application}{
		e_1 \transforms e_1' \\ e_2 \transforms e_2'
	}{
		\Gamma \vdash e_1 e_2 : \tau \transforms e_1' e_2'	
	}
\end{mathpar}

\begin{mathpar}
	\relationRule{Let-binding-tuple}{
          e_1 \transforms e_1' \\ e_2 \transforms (e_{21},\ldots,e_{2n})
	}{
		\Gamma \vdash \ttt{let}\ p = e_1\ \ttt{in}\ e_2 : \ttuple \transforms 
                (\ttt{let}\ p = \ e_1' \ttt{in}\ e_{21},\ldots,\ttt{let}\ p = \ e_1' \ttt{in}\ e_{2n})
	}
\end{mathpar}

\begin{mathpar}
	\relationRule{Let-binding}{
          \tau \neq \ttuple \\ e_1 \transforms e_1' \\ e_2 \transforms e_2'
	}{
		\Gamma \vdash \ttt{let}\ p = e_1\ \ttt{in}\ e_2 : \tau \transforms \ttt{let}\ p = \ e_1' \ttt{in}\ e_2'	
	}
\end{mathpar}

\begin{mathpar}
	\relationRule{Match-Tuple}{
          e \transforms e' \\ e_1 \transforms (e_{11},\ldots,e_{1n}) \\ \ldots \\ (e_{k1},\ldots,e_{kn})
	}{
		\Gamma \vdash \ttt{match}\ e\ \ttt{with}\ |\ p_1 \to e_1 \dots |\ p_k \to e_k : \ttuple \transforms 
		(\ttt{match}\ e'\ \ttt{with}\ |\ p_1 \to e_{11} \dots |\ p_k \to e_{k1}, \ldots,
                \ttt{match}\ e'\ \ttt{with}\ |\ p_1 \to e_{1n} \dots |\ p_k \to e_{kn})
	}
\end{mathpar}

\begin{mathpar}
	\relationRule{Match}{
          \tau \neq \ttuple \\ e \transforms e' \\ e_1
	}{
          \Gamma \vdash \ttt{match}\ e\ \ttt{with}\ |\ p_1 \to e_1 \dots |\ p_k \to e_k : \tau = 
		\ttt{match}\ e'\ \ttt{with}\ |\ p_1 \to e_1' \dots |\ p_k \to \rel{e_k}
	}
\end{mathpar}

\begin{mathpar}
	\relationRule{If-else}{
		\null
	}{
		\rel{\ttt{if}\ e_1\ \ttt{then}\ e_2\ \ttt{else}\ e_3} = 
		\ttt{if}\ \rel{e_1}\ \ttt{then}\ \rel{e_2}\ \ttt{else}\ \rel{e_3}
	}
\end{mathpar}

\section*{Declarations}

\begin{mathpar}
	\relationRule{Let declaration}{
		\null
	}{
		\rel{\ttt{let}\ x = e} = \ttt{let}\ x = \rel{e}
	}
\end{mathpar}

\begin{mathpar}
	\relationRule{Attribute type - not $\tau_m$}{
		\tau \neq \tau_m
	}{
		\rel{\ttt{type attribute} = \tau} = \ttt{type attribute} = \tau
	}
\end{mathpar}


\begin{mathpar}
	\relationRule{Attribute type - $\tau_m$}{
		\null
	}{
		\rel{\ttt{type attribute} = \tau_m} = \ttt{type attribute} = \tau_t
	}
\end{mathpar}

\begin{mathpar}
	\relationRule{Symbolic type - not $\tau_m$}{
		\tau \neq \tau_m
	}{
		\rel{\ttt{symbolic}\ x : \tau} = \ttt{symbolic}\ x : \tau
	}
\end{mathpar}

\begin{mathpar}
	\relationRule{Symbolic type - $\tau_m$}{
		\null
	}{
		\rel{\ttt{symbolic}\ x : \tau_m} = \ttt{symbolic}\ x : \tau_t
	}
\end{mathpar}

\begin{mathpar}
	\relationRule{Symbolic value}{
		\null
	}{
		\rel{\ttt{symbolic}\ x = e} = \ttt{symbolic}\ x = \rel{e}
	}
\end{mathpar}

\begin{mathpar}
	\relationRule{Require}{
		\null
	}{
		\rel{\ttt{require}\ e} = \ttt{require}\ \rel{e}
	}
\end{mathpar}

\begin{mathpar}
	\relationRule{Node declaration}{
		\null
	}{
		\rel{\ttt{let nodes} = n} = \ttt{let nodes}\ = n
	}
\end{mathpar}

\begin{mathpar}
	\relationRule{Edge declaration}{
		\null
	}{
		\rel{\ttt{let edges} = lst} = \ttt{let edges}\ = lst
	}
\end{mathpar}

\section*{Map expressions}
Each of the following expressions have at least one subexpression which must be a map. The following rules apply if and only if those subexpressions have type $\tau_m$. If not, we recurse as usual but don't otherwise change the expression.

I'm noting that some of these rules involve a lot of duplication, so we might want to look into translations
which e.g. store duplicated expressions in a local variable first.

They also involve a lot of extra unpacking since we don't have syntax for getting/setting with tuples. It
would probably be good to extend the map syntax to work on tuples as well, so long as that doesn't mess
anything up; we could also create a new syntax for tuples.

\begin{mathpar}
	\relationRule{CreateDict}{
		e' = \rel{e}
	}{
		\rel{\ttt{createDict}\ e} = (e', \dots, e')	
	}
\end{mathpar}

\begin{mathpar}
	\relationRule{Get}{
		k = k_i \\
		x_i \mbox{ fresh}
	}{
		\rel{m[k_i]} = 
		\ttt{let}\ (\_, \dots, x_i, \dots, \_) = \rel{m}\ \ttt{in}\ 
		x_i
	}
\end{mathpar}

\begin{mathpar}
	\relationRule{Set - known key}{
		k = k_i \\
		x_1, \dots, x_n \mbox{ fresh}
	}{
		\rel{m[k := e]} = 
		\ttt{let}\ (x_1, \dots, x_n) = \rel{m}\ \ttt{in}\ 
		(x_1, \dots, \rel{e}, \dots, x_n)
	}
\end{mathpar}

\begin{mathpar}
	\relationRule{Set - unknown key}{
		\forall i.k \neq k_i \\
		x_1, \dots, x_n \mbox{ fresh}
	}{
		\rel{m[k := e]} = \rel{m}
	}
\end{mathpar}

\begin{mathpar}
	\relationRule{Map}{
		f' = \rel{f} \\
		x_1, \dots, x_n \mbox{ fresh}
	}{
		\rel{\ttt{map}\ f\ m} = 
		\ttt{let}\ (x_1, \dots, x_n) = \rel{m}\ \ttt{in}\ 
		(f'\ x_1, \dots, f'\ x_n)
	}
\end{mathpar}

\begin{mathpar}
	\relationRule{MapIf}{
		f' = \ttt{fun}\ (k, v)\ \to \ttt{if}\ p\ k\ \ttt{then}\ \rel{f}\ v\ \ttt{else}\ v \\
		x_1, \dots, x_n \mbox{ fresh}
	}{
		\rel{\ttt{mapIf}\ p\ f\ m} = 
		\ttt{let}\ (x_1, \dots, x_n) = \rel{m}\ \ttt{in}\ 
		(f'\ (k_1, x_1), \dots, f'\ (k_n, x_n))
	}
\end{mathpar}

\begin{mathpar}
	\relationRule{Filter}{
		p' = \ttt{fun}\ (k, v)\ \to \ttt{if}\ p\ k\ \ttt{then}\ v\ \ttt{else}\ \ttt{false} \\
		x_1, \dots, x_n \mbox{ fresh}
	}{
		\rel{\ttt{filter}\ p\ m} = 
		\ttt{let}\ (x_1, \dots, x_n) = \rel{m}\ \ttt{in}\ 
		(p'\ (k_1, x_1), \dots, p'\ (k_n, x_n))
	}
\end{mathpar}

\begin{mathpar}
	\relationRule{Combine}{
		f' = \rel{f} \\
		x_1, \dots, x_n, y_1, \dots y_n \mbox{ fresh}
	}{
		\rel{\ttt{combine}\ f\ m_1\ m_2} = 
		\ttt{let}\ ( x_1, \dots, x_n) = \rel{m_1}\ \ttt{in}\ 
		\ttt {let}\ (y_1, \dots,  y_n) = \rel{m_2}\ \ttt{in}\ \\ 
		(f'\ x_1\ y_1, \dots, f'\ x_n\ y_n))
	}
\end{mathpar}

\begin{mathpar}
	\relationRule{Union}{
		f' = \rel{f} \\
		x_1, \dots, x_n, y_1, \dots y_n \mbox{ fresh}
	}{
		\rel{m_1\ \ttt{union}\ m_2} = 
		\ttt{let}\ (x_1, \dots, x_n) = \rel{m_1}\ \ttt{in}\ 
		\ttt{let}\ (y_1, \dots, y_n) = \rel{m_2}\ \ttt{in}\
		(x_1\ \|\ y_1, \dots, x_n\ \|\ y_n)
	}
\end{mathpar}


\begin{mathpar}
	\relationRule{Intersection}{
		f' = \rel{f} \\
		x_1, \dots, x_n, y_1, \dots y_n \mbox{ fresh}
	}{
		\rel{m_1\ \ttt{inter}\ m_2} = 
		\ttt{let}\ (x_1, \dots, x_n) = \rel{m_1}\ \ttt{in}\ 
		\ttt{let}\ (y_1, \dots, y_n) = \rel{m_2}\ \ttt{in}\
		(x_1\ \&\&\ y_1, \dots, x_n\ \&\&\ y_n)
	}
\end{mathpar}

\end{document}