\documentclass{article}

\usepackage{fancyhdr}
\usepackage{extramarks}
\usepackage{amsfonts}
\usepackage{syntax}
\usepackage{stmaryrd}
\usepackage{mathpartir}

%
% Basic Document Settings
%

\topmargin=-0.45in
\evensidemargin=0in
\oddsidemargin=0in
\textwidth=6.5in
\textheight=9.0in
\headsep=0.25in

\linespread{1.1}

\pagestyle{fancy}
\chead{Tuple flattening}
\lfoot{\lastxmark}
\cfoot{\thepage}

\renewcommand\headrulewidth{0.4pt}
\renewcommand\footrulewidth{0.4pt}

\setlength\parindent{30pt}

\newcommand{\Z}{\mathbb{Z}}
\newcommand{\Zt}{$\Z$}

% Create a relational rule
% [#1] - Additional mathpartir arguments
% {#2} - Name of the rule
% {#3} - Premises for the rule
% {#4} - Conclusions for the rule
\newcommand{\relationRule}[4][]{\inferrule*[lab={\sc #2},#1]{#3}{#4}}

\newcommand{\rel}[1]{\ensuremath{\llbracket {#1} \rrbracket}}
\newcommand{\ttt}{\texttt}
\newcommand{\transform}{\rightsquigarrow}
\newcommand{\proj}{\pi}
\newcommand{\ttuple}{(\tau_1, \ldots, \tau_n)}
\newcommand{\etuple}{(e_1,\ldots,e_n)}
\newcommand{\bool}{\mathrm{bool}}
\newcommand{\integer}{\mathrm{int}}
\newcommand{\option}{\mathrm{option}}
\newcommand{\uoption}[1]{(\bool,#1)}
\newcommand{\opt}[1]{\texttt{opt-#1}}
\newcommand{\varOf}[1]{\texttt{varOf}(#1)}



\begin{document}
This document defines a transformation $\transform$ that unboxes options as a tuple between a boolean and the value it (may) hold. 


\section*{Types}

\begin{mathpar}
	\relationRule{basic type}{
          \tau \in \{\bool,\integer\}
	}{
          \tau \transform \tau
	}

	\relationRule{option type}{
          \tau \transform \tau'
	}{
          \option~ \tau \transform \option{\tau'}
	}

	\relationRule{tuple type}{
          \tau_1 \transform \tau_1' \\ \ldots \\ \tau_n \transform \tau_n'
	}{
          \ttuple \transform \tau_1 * \ldots * \tau_n
	}

	\relationRule{arrow type}{
          \tau_1 \transform \tau_1' \\ \tau_2 \transform \tau_2'
	}{
          \tau_1 \to \tau_2 \transform \tau_1' \to \tau_2'
	}
\end{mathpar}

\section*{Values}

\begin{mathpar}
	\relationRule{literal}{
		\tau \in \{\texttt{bool}, \texttt{int}\}
	}{
           n : \tau \transform n
	}
\end{mathpar}

\begin{mathpar}
	\relationRule{None}{
          \tau \transform \tau'
	}{
           \texttt{None} : \option~ \tau \transform \texttt{None} : \option \tau'
	}
\end{mathpar}

\begin{mathpar}
  \relationRule{Function}{
    \tau_1 \transform \tau_1' e : \tau_2 \transform e' : \tau_2'
  }
{\texttt{fun}\ x \to e : \tau_1 \to \tau_2 \transform 
 \texttt{fun}\ x \to e' : \tau_1' \to \tau_2'
}
\end{mathpar}

\section*{Expressions}

\begin{mathpar}
	\relationRule{identifer }{
          \tau \transform \tau'
	}{
          x : \tau \transform x : \tau'
	}
\end{mathpar}

\begin{mathpar}
	\relationRule{Tuple}{
		e_1 : \tau_1 \transform e_1' : \tau_1' \\ \ldots \\ 
                e_n : \tau_n \transform e_n' : \tau_1'
	}{
          \etuple : \ttuple \transform (e_1,\ldots,e_n) : \tau_1' * \ldots * \tau_n'
	}
\end{mathpar}

\begin{mathpar}
	\relationRule{Not}{
          e : \tau \transform e' : \tau'
	}{
		!e : \tau \transform !e' : \tau'	
	}
\end{mathpar}
	
\begin{mathpar}
\relationRule{Binop}{
	 \oplus \mbox{ is a binary operator} \\ 
         e_1 : \tau_1 \transform e_1' : \tau_1' \\ e_2 : \tau_2 \transform e_2' : \tau_2'
         \tau \transform \tau'
   }{
   	e_1 \oplus e_2 : \tau \transform e_1' \oplus e_2' : \tau'
   }
\end{mathpar}

\begin{mathpar}
	\relationRule{Function application}{
		e_1 : \tau_1 \to \tau_2 \transform e_1' : \tau_1' \to \tau_2' \\ 
                e_2 : \tau_2 \transform e_2' : \tau_2'
	}{
                e_1 e_2 : \tau \transform e_1' e_2' : \tau2'	
	}
\end{mathpar}


\begin{mathpar}
	\relationRule{Let-binding}{
          e_1 : \tau_1 \transform e_1' : \tau_1' \\ e_2 : \tau_2 \transform e_2' : \tau_2'
	}{
          \texttt{let}\ p = e_1\ \texttt{in}\ e_2 : \tau_2 \transform \texttt{let}\ p = \ e_1' \texttt{in}\ e_2' : \tau_2'	
	}
\end{mathpar}

\begin{mathpar}
	\relationRule{Match}{
          e \transform e' \\ e_1 \transform e_1' \\ \ldots \\ e_2 \transform e_2' \\
          p_1 \transform p_1' \\ \ldots \\ p_k \transform p_k'
	}{
		\texttt{match}\ e\ \texttt{with}\ |\ p_1 \to e_1 \dots |\ p_k \to e_k : \ttuple \transform 
		\texttt{match}\ e'\ \texttt{with}\ |\ p_1' \to e_1' \dots |\ p_k' \to e_k' 
	}
\end{mathpar}

\begin{mathpar}
	\relationRule{If-else}{
		e_1 \transform e_1' \\ e_2 \transform e_2' \\ e_3 \transform e_3'
	}{
		\texttt{if}\ e_1\ \texttt{then}\ e_2\ \texttt{else}\ e_3 = 
		\texttt{if}\ e_1'\ \texttt{then}\ e_2'\ \texttt{else}\ e_3'
	}
\end{mathpar}


\section*{Patterns}

\begin{mathpar}
	\relationRule{PNone}{
          \null
	}{
		\texttt{None} \transform (\texttt{false}, \_)	
	}
\end{mathpar}

\begin{mathpar}
	\relationRule{PSome}{
          p \transform p'
	}{
		\texttt{Some}~p \transform (\texttt{true}, p')	
	}
\end{mathpar}

\begin{mathpar}
	\relationRule{PTuple}{
          p_1 \transform p_1' \\ p_n \transform p_n'
	}{
		(p_1,\ldots,p_n) \transform (p_1',\ldots,p_n')
	}
\end{mathpar}

\begin{mathpar}
	\relationRule{Pother}{
          p = \_ \vee p = x \vee p = b : bool \vee p = n : int
	}{
          p \transform p
	}
\end{mathpar}

\end{document}
%%% Local Variables:
%%% mode: latex
%%% TeX-master: t
%%% End:
