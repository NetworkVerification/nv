\documentclass[sigconf,10pt]{acmart}
\usepackage{amsmath}           % basic math features
\usepackage{amssymb}           % basic additional symbols
\usepackage{accents}           % mathematicsl accents like hat, vector etc.
\usepackage{xcolor}            % create and use custom colors (e.g., for highlighting)
\usepackage{xspace}            % adds a space if needed at the end of a macro
\usepackage{balance}           % \balance to the last page makes refs symmetric
\usepackage{enumitem}          % allows customizing lists
\usepackage[utf8]{inputenc}    % document encoding set to utf8
\usepackage{graphicx}          % better includegraphics command
\usepackage{etoolbox}          % enable patching commands, for below
\patchcmd{\quote}{\rightmargin}{\leftmargin 1em \rightmargin}{}{}
\usepackage{amsthm}            % basic theorems
\usepackage{thmtools}          % more flexible theorems
\usepackage{thm-restate}       % repeat theorems in appendix if needed
\usepackage{hyperref}          % hyper-text links
\usepackage{cleveref}          % smarter type-based cross-references
\usepackage{apptools}          % customizable appendix
\usepackage{local}             % user-defined commands stored in local.sty

% break long urls in the bibliography
\def\UrlBreaks{\do\/\do-}
\hypersetup{breaklinks=true}
%\hypersetup{draft}

% reclaim the copyright space for the draft
\settopmatter{printacmref=false}
\renewcommand\footnotetextcopyrightpermission[1]{}
\pagestyle{plain}

\begin{document}
\title{NV: An intermediate language for network verification}


\maketitle

%=====================================================
%
%
%  **Abstract**
%
%
%=====================================================

\textbf{Abstract---}
NV is great

%=====================================================
%
%
%  **Introduction**
%
%
%=====================================================

\section{Introduction} 
\label{sec:introduction}

Stuff

%=====================================================
%
%
%  **Overview of our approach**
%
%
%=====================================================

\section{Motivating Examples} 
\label{sec:motivation}

Key ideas

%=====================================================
%
%
%  **Language and Semantics**
%
%
%=====================================================

\section{Language} 
\label{sec:language}

\begin{enumerate}
  \item Syntax and overview
  \item Features, \EG, symbolic variables, requires, types, fixed point etc.
  \item Semantics: (1) Asychronous semantics (simulation), and (2) Stable path semantics (SMT)
\end{enumerate}

%=====================================================
%
%
%  **Tools**
%
%
%=====================================================

\section{Tools}
\label{sec:tools}

\begin{enumerate}
  \item Simulator
  \item SMT encoding (stable)
  \item SMT encoding (bounded model checker)
  \item Randomized testing / Fuzzing?
  \item Symbolic execution 
  \item Abstraction / symmetries
  \item Abstraction refinement for failures
  \item Compiler to P4
  \item Convergence checking?
\end{enumerate}

%=====================================================
%
%
%  **Evaluation**
%
%
%=====================================================

\section{Evaluation}
\label{sec:evaluation}

\begin{enumerate}
  \item comparison to Batfish?
  \item run the tool on real networks to find bugs?
\end{enumerate}

%=====================================================
%
%
%  **Related work**
%
%
%=====================================================

\section{Related Work}
\label{sec:related}


%=====================================================
%
%
%  **Conclusion**
%
%
%=====================================================

\section{Conclusion}
\label{sec:conclusions}


%=====================================================
%
%
%  **Bibliography**
%
%
%=====================================================

%\balance
 
\bibliographystyle{abbrv}
\bibliography{references}

\end{document}